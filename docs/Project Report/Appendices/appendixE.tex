\chapter{REAMDME.md Configuration File}

\section{Building the project}\label{building-the-project}

This directory contains some sample config files that can be used a
reference for setting up the project. Feel free to use them and change
the variables accordingly.

\subsection{Setting up Apache2}\label{setting-up-apache2}

\begin{enumerate}
\def\labelenumi{\arabic{enumi}.}
\item
  \texttt{sudo apt install apache2 -y} - Install apache2
\item
  \texttt{sudo apt install libapache2-mod-auth-openidc} - Installs
  library for running openid
\item
  Enable the following mods: \texttt{ssl}, \texttt{auth\_openidc},
  \texttt{wsgi}, \texttt{rewrite}
\item
  \texttt{sudo nano /etc/apache2/auth\_openidc.conf} and copy the
  following to the file
\end{enumerate}

\begin{lstlisting}[language=bash]
OIDCProviderMetadataURL https://openidc.dcs.aber.ac.uk/auth/realms/MMP-IMPACS/.well-known/openid-configuration
OIDCClientID MMP-IMPACS 
OIDCRedirectURI /oauth2callback #Replace with link to website e.g. "discord.dcs.aber.ac.uk/oauth2callback"
OIDCCryptoPassphrase some-random-string-for-encrypting-cookies
OIDCScope "openid basic"
OIDCRemoteUserClaim preferred_username
OIDCSessionInactivityTimeout 86400
\end{lstlisting}

\begin{enumerate}
\def\labelenumi{\arabic{enumi}.}
\setcounter{enumi}{4}
\item
  \texttt{sudo cp \textasciitilde{}/aberlink/config/aberlink.conf /etc/apache2/sites-available/}
  - copy the config file to the directory
\item
  Edit the file to match the file structure, below is an example of what
  to change the settings to:
\end{enumerate}

\begin{lstlisting}[language=bash]
<VirtualHost *:80>
        ServerName discord.dcs.aber.ac.uk
        ServerAlias discord.dcs.aber.ac.uk
        ServerAdmin cs-support@aber.ac.uk
        DocumentRoot /home/joel/aberlink/src/

        ErrorLog ${APACHE_LOG_DIR}/error.log
        CustomLog ${APACHE_LOG_DIR}/access.log combined
</VirtualHost>
\end{lstlisting}

\begin{enumerate}
\def\labelenumi{\arabic{enumi}.}
\setcounter{enumi}{6}
\item
  \texttt{sudo a2ensite aberlink} - Enable the website
\item
  Go to \texttt{sudo nano /etc/apache2/apache2.conf} and add the
  following:
\end{enumerate}

\begin{lstlisting}[language=bash]
<Directory /home/joa38/aberlink/src/>
        Options Indexes FollowSymLinks
        AllowOverride None
        Require all granted
</Directory>
\end{lstlisting}

\subsection{Setting up SSL}\label{setting-up-ssl}

\begin{enumerate}
\def\labelenumi{\arabic{enumi}.}
\item
  Install the certbot package: \texttt{sudo apt intall certbot}
\item
  Run this command to create the SSL certificates:
  \texttt{sudo certbot certonly -{}-webroot -w \textasciitilde{}/aberink -d discord.dcs.aber.ac.uk}
\item
  \texttt{sudo a2enmod rewrite} - Enable the rewrite mod for website
  redirect
\item
  \texttt{sudo a2enmod ssl} - Enable the ssl mod for SSL to work on the
  website
\item
  \texttt{source /etc/apache2/envvars} - fixes errors in apache2
\item
  Open the conf file again (e.g. \texttt{aberlink.conf}) and change it
  match the domain name:
\end{enumerate}

\begin{lstlisting}[language=bash]
<VirtualHost *:443>
        ServerAdmin cs-support@aber.ac.uk
        serverName discord.dcs.aber.ac.uk
        ServerAlias discord.dcs.aber.ac.uk
        DocumentRoot /home/joa38/aberlink/src/AberLinkAuthentication

        ErrorLog ${APACHE_LOG_DIR}/error.log
        CustomLog ${APACHE_LOG_DIR}/access.log combined

        Alias /static /home/joa38/aberlink/src/AberLinkAuthentication/static
        <Directory /home/joa38/aberlink/src/AberLinkAuthent/static>
                Require all granted
        </Directory>

        <Directory /home/joa38/aberlink/src/AberLinkAuthentication/AberLinkAuthentication>
                <Files wsgi.py>
                        Require all granted
                </Files>
        </Directory>

        <Directory /home/joa38/aberlink/src/AberLinkAuthentication>
                Options Indexes FollowSymLinks
                AllowOverride None
                Require all granted
                Allow from all
        </Directory>

        <Location />
                AuthType openid-connect
                Require valid-user
        </Location>

        WSGIScriptAlias / /home/joa38/aberlink/src/AberLinkAuthentication/AberLinkAuthentication/wsgi.py
        WSGIDaemonProcess django_app python-path=/home/joa38/aberlink/src/AberLinkAuthentication python-home=/home/joa38/aberlink/src/AberLinkAuthentication/venv
        WSGIProcessGroup django_app

        SSLEngine on
        SSLCertificateFile /etc/letsencrypt/live/discord.dcs.aber.ac.uk/fullchain.pem
        SSLCertificateKeyFile /etc/letsencrypt/live/discord.dcs.aber.ac.uk/privkey.pem
        SSLProtocol all -SSLv2 -SSLv3 -TLSv1 -TLSv1.1
</VirtualHost>
\end{lstlisting}

\begin{enumerate}
\def\labelenumi{\arabic{enumi}.}
\setcounter{enumi}{6}
\itemsep1pt\parskip0pt\parsep0pt
\item
  \texttt{sudo apachectl restart} - Restart Apache2 to enable the new
  mods
\end{enumerate}

\subsection{installing the dependencies for the
bot}\label{installing-the-dependencies-for-the-bot}

\begin{enumerate}
\def\labelenumi{\arabic{enumi}.}
\item
  \texttt{sudo apt install libpq-dev python-dev} - needed to install
  \texttt{psycopg2-binary}
\item
  Navigate to the folder \texttt{aberlink/src/AberLinkDiscord} and type
  \texttt{pipenv install}
\item
  Create a discord bot token by visitng
  \url{https://discord.com/developers/applications} and creating a new
  bot called AberLink along with the supplied photo
  \texttt{/img/AberLink\_logo\_cropped.png}. Then head on over to the
  Bot panel and create a new bot with the same information as above.
  Finally copy the token underneath the bots name for the next section.
\item
  While inside the folder \texttt{/aberlink/src/AberLinkDiscord} create
  a new file using \texttt{nano .env} and add the following (or copy the
  example provided in this folder):
\end{enumerate}

\begin{lstlisting}[language=bash]
DISCORD_TOKEN= # Discord token found on bot's page
DATABASE_NAME= # Postgres database name
USER= # Postgres user
PASSWORD= # Postgres password
HOST= # Postgres host
PORT=5432 # This is the default port
WEBSITE_URL= # The website for AberLinkAuthentication e.g. https://joa38-mmp.dcs.aber.ac.uk/ 
\end{lstlisting}

\begin{enumerate}
\def\labelenumi{\arabic{enumi}.}
\setcounter{enumi}{4}
\item
  Check that the .env file has been configured correctly by typing in
  the command \texttt{pipenv run python3 -m AberLink}.
\item
  Invite the bot to one of your servers by creating an invite link in
  the tab OAuth2 and selecting \texttt{bot} in the \texttt{scopes}
  section and then in the \texttt{bot permissions} tick
  \texttt{Administrator}. Copy the link generated by it and paste it
  into the browser and invite the bot to your server.
\item
  In the OAuth2 tab \texttt{Redirects} section add your website uri e.g.
  \url{https://discord.dcs.aber.ac.uk/oauth2/login/redirect}.
\item
  On the same page scroll down to the \texttt{OAuth2 URL Generator} and
  select the web address you entered in the \texttt{Redirects} section.
\item
  Scroll down again and in the \texttt{scopes} section select
  \texttt{identify} and copy the url. Then head on over to the file
  \texttt{/aberlink/src/AberLinkAuthentication/login/views.py} and find
  the \texttt{discord\_oauth2} function and replace the redirect URL
  with your own.
\end{enumerate}

\subsection{Installing and running the
webserver}\label{installing-and-running-the-webserver}

\begin{enumerate}
\def\labelenumi{\arabic{enumi}.}
\item
  navigate to the folder \texttt{aberlink/src/AberLinkAuthentication}
  and type \texttt{virtualenv venv} to create a virtual enviornment
  folder for the data.
\item
  \texttt{source venv/bin/activate} - activates the virtuanenv
\item
  \texttt{pipenv install} installs dependencies from the project into
  the file
\item
  Make sure that that the venv file path is correct inside of the
  \texttt{aberlink.conf} file.
\end{enumerate}

\subsection{Django setup}\label{django-setup}

\begin{enumerate}
\def\labelenumi{\arabic{enumi}.}
\itemsep1pt\parskip0pt\parsep0pt
\item
  \texttt{sudo cp config/config.json /etc/config.json} - copy the
  template file for the django config and fill out the details below
  (email joa38@aber.ac.uk for \texttt{SECRET\_KEY}):
\end{enumerate}

\begin{lstlisting}[language=json]
{
    "SECRET_KEY": "",
    "DATABASE_NAME": "",
    "USER": "",
    "PASSWORD": "",
    "HOST": "",
    "PORT": "",
    "DISCORD_CLIENT_SECRET": "",
    "DISCORD_TOKEN": "",
    "WEBSITE_URL": ""
}
\end{lstlisting}
Note: The \texttt{WEBSITE\_URL} is the name of the website that is going
to be used. e.g. \texttt{https://joa38-mmp.dcs.aber.ac.uk}

The \texttt{DISCORD\_CLIENT\_SECRET} and \texttt{DISCORD\_TOKEN} can be
found by visitng the page created earlier for the discord bot.

\begin{enumerate}
\def\labelenumi{\arabic{enumi}.}
\item
  Navigate to the \texttt{General Information} tab and copy the client
  secret which is located below the Description on the right hand side.
  Copy and save this variable to the \texttt{DISCORD\_TOKEN}
\item
  Naviage to the \texttt{Bot} tab and copy the name token located below
  the username and save it to the \texttt{DISCORD\_CLIENT\_SECRET}
\end{enumerate}

After setting up the config file open the command shell and type the
following commands to configure the database for Django:

\begin{enumerate}
\def\labelenumi{\arabic{enumi}.}
\item
  \texttt{python3 manage.py makemigrations}
\item
  \texttt{python3 manage.py sqlmigrate}
\item
  \texttt{python3 manage.py migrate}
\end{enumerate}

Finally once everything has been ensured to be fully functioning open
the file \\\verb|src/AberLinkAuthentication/AberLinkAuthentication/settings.py| and find the line \texttt{DEBUG = True} and set the variable to
\texttt{False}
