\chapter{Testing}

%Detailed descriptions of every test case are definitely not what is required here. What is important is to show that you adopted a sensible strategy that was, in principle, capable of testing the system adequately even if you did not have the time to test the system fully.

%Provide information in the body of your report and the appendix to explain the testing that has been performed. How does this testing address the requirements and design for the project?

%How comprehensive is the testing within the constraints of the project?  Are you testing the normal working behaviour? Are you testing the exceptional behaviour, e.g. error conditions? Are you testing security issues if they are relevant for your project? 

%Have you tested your system on ``real users''? For example, if your system is supposed to solve a problem for a business, then it would be appropriate to present your approach to involve the users in the testing process and to record the results that you obtained. Depending on the level of detail, it is likely that you would put any detailed results in an appendix.

%Whilst testing with ``real users'' can be useful, don't see it as a way to shortcut detailed testing of your own. Think about issues discussed in the lectures about until testing, integration testing, etc. User testing without sensible testing of your own is not a useful activity.

%The following sections indicate some areas you might include. Other sections may be more appropriate to your project.   

As previously mentioned in section \ref{sec1:pro} a Feature Driven Development (FDD) approach was used for this project, working in one week iteration windows to build and test the software. This included all of the testing sections mentioned below apart from automated testing which was completed at the end of the project instead.

The Discord.py Python framework does not contain any unit testing and there are no third party libraries which sufficiently perform this job. For the Django framework however there are some unit tests built in so they have been used in this project and are detailed further on in section \ref{sec4:unit-web}. These Django tests are used to cover both the website and the database as the website cannot be tested by itself. 

\section{Sample Data}
Sample data was considered for this project however it would be very difficult to implement because the OpenID Connect \cite{OpenID} system used to authenticate and log users in relies on a proper database to pull user data from. The university was not comfortable with creating and supplying fake user accounts that would authenticate with this system as the database where these would be used is currently live. This would mean that they could be stolen and used to access live systems such as Student Record or blackboard. The alternative to this would mean that a mock database would be needed which would take up a considerable work to create. The workaround for this system is far simpler and involved asking both staff and students to try out the system and test it until it breaks. The outcomes of this are detailed in section \ref{sec4:user-tesing}.

\section{Automated Testing}
Automated testing has been difficult for this project as mentioned above. Originally there was a plan to include automated testing by using the DevOps frameworks in Git, however as the universities GitLab instance was used then there was no access to dedicated pipelines/kubernite clusters to build and run the code.  

\subsection{Unit Tests}

\subsubsection{Website \& Database}\label{sec4:unit-web}
Django provides some useful documentation (\href{https://docs.djangoproject.com/en/3.1/topics/testing/}{https://docs.djangoproject.com/en/3.1/topics/testing/}) for creating unit tests for both Django and the database/user models. These tests are useful for checking that Discord accounts and OpenID Connect \cite{OpenID} Aber accounts link up properly and that webpage urls are correct and render the appropriate content. All of the following tests can be found in the file located in this path \verb|src\AberLinkAuthentication\login\tests.py|. 

Below are two testing classes; one called TestUrls and one called TestModels. These are responsible for testing that the urls are correctly found and that adding new users to the database works as intended. Included below is an example of each test:

\begin{figure}[H]
\begin{lstlisting}[language=Python]
def test_url_discord_redirect(self):
    url = reverse('Discord-response')
    self.assertEquals(resolve(url).func, views.discord_oauth2_redirect)
\end{lstlisting}
\caption{Django URL render test}
\label{fig:django-url}
\end{figure}
As seen above the test function simply gets the name of the url 'Discord-response' and then checks that it returns the correct Python function (view) to render that page

\begin{figure}[H]
\begin{lstlisting}[language=Python]
def test_model_openidc_user_staff(self):
    self.openidc_user2 = OpenIDCUser.objects.create(
        username="abc123",
        name="Bob Ross",
        email="abc123@aber.ac.uk",
        usertype="staff"
    )
    self.assertEquals(self.openidc_user2.username, "abc123")
    self.assertEquals(self.openidc_user2.name, "Bob Ross")
    self.assertEquals(self.openidc_user2.email, 
    "abc123@aber.ac.uk")
    self.assertEquals(self.openidc_user2.usertype, "staff")
    self.assertEquals(self.openidc_user2.is_admin, True)
\end{lstlisting}
\caption{Django database render test}
\label{fig:django-database}
\end{figure}
This function creates a new user object and then checks to make sure that the user has been given the admin and has the usertype of staff.

\subsubsection{Discord}
Unit testing in Discord.py is impossible as there is no included framework for it and there are no external libraries capable of testing to check that it works.

\subsection{Stress Testing}
To test the websites capacity to deal with 100s of requests ApacheBenchmark \cite{apacheBenchmark} is used which is a built into the Apache2 \cite{apache2} and is a module used to request a webpage and measure response time. To test this the OpenID Connect \cite{OpenID} authentication system  was temporarily disabled so that HTML gets served to the incoming requests instead of being redirected to the login page. Due to the nature of the websites implementation you cannot query the main page because the user is not authenticated and the page can't display information on the user, however it can query one of the subpages that does not require authentication. The following line of code was used to complete the request with 1000 requests and 100 concurrent threads to attempt to tank the website's performance.

\verb|ab -n 1000 -c 100 https://mmp-joa38.dcs.aber.ac.uk/privacy-policy|

This command generates the following report which shows that the website is capable at easily handling this many requests simultaneously. The longest response time for serving the requests was 642ms which is still adequately fast for the average user and shows that Django handles requests reasonable well. It can also serve on average 174 requests per second and I expect that the website will never be receiving more that 300 requests at the exact same time which can definitely be handled as seen from these results. Another factor to consider is that this development build is currently running on a small container that only has access to a small amount of resources whereas in the final build it will have access to far more resources so could serve requests up even quicker.

%\begin{figure}[hbt!]
\begin{verbatim}
Server Software:        Apache/2.4.38
Server Hostname:        mmp-joa38.dcs.aber.ac.uk
Server Port:            443
SSL/TLS Protocol:       TLSv1.2,ECDHE-RSA-AES256-GCM-SHA384,2048,256
Server Temp Key:        X25519 253 bits
TLS Server Name:        mmp-joa38.dcs.aber.ac.uk

Document Path:          /privacy-policy
Document Length:        12448 bytes

Concurrency Level:      100
Time taken for tests:   5.719 seconds
Complete requests:      1000
Failed requests:        0
Total transferred:      12734000 bytes
HTML transferred:       12448000 bytes
Requests per second:    174.86 [#/sec] (mean)
Time per request:       571.882 [ms] (mean)
Time per request:       5.719 [ms] (mean, across all concurrent requests)
Transfer rate:          2174.50 [Kbytes/sec] received

Connection Times (ms)
              min  mean[+/-sd] median   max
Connect:        9  422 105.1    458     547
Processing:     7   86  61.1     67     404
Waiting:        6   83  60.1     64     381
Total:         42  508  96.3    531     642

Percentage of the requests served within a certain time (ms)
  50%    531
  66%    550
  75%    560
  80%    565
  90%    574
  95%    583
  98%    593
  99%    602
 100%    642 (longest request)
\end{verbatim}
%\caption{Output from running Apache Benchmark on the website}
%\label{fig:ab}
%\end{figure}

\section{User Interface Testing}
Testing the user interface for the AberLink is mostly straight forward as Discord has full control over the UI so I've had no testing to do on that front.

The UI of the website uses the library Bootstrap \cite{bootstrap} as it is built from the ground up with responsive design in mind and saves a lot of time that would be required to re-learn CSS fully. This meant that website would scale very nicely and uniformly depending on screen size. To ensure that the website scales and performs well the website and all of its content have been tested on multiple devices. It has been tested on mobile devices, iPads and laptops with screens ranging from 11in to 40in using Safari, Chrome, Edge and Firefox. The website has also been tested on HTML the validation website \href{https://validator.w3.org/}{https://validator.w3.org/} to ensure that there are no large errors that may cause the website to not load on certain machines.

\section{User Testing}\label{sec4:user-tesing}
Throughout this project I have worked with the mindset that you develop small sections of code and then review what effect they have on the system. This helps to catch errors as it is much easier to backtrack through small sections of code.

For user testing I asked in the comp sci server if students could try and login to the website and break it. This turned out to work very well and helped to find a few edge cases that I had previously missed in the code.

\subsection{Attendance API Endpoint Testing}
The supplied API for attendance has been tested using user testing and the university provided me with 3 different endpoints to test. They are listed below along their responses.

\begin{verbatim}
https://integration.aber.ac.uk/joa38/good.php
{"status_updated":"true","request":"200","module_code":"CS32120"}    
\end{verbatim}

\begin{verbatim}
https://integration.aber.ac.uk/joa38/bad.php
{"status_updated":"false","request":"400",
"error_message":"Unknown user or no event"}
\end{verbatim}

\begin{verbatim}
https://integration.aber.ac.uk/joa38/submit.php
Actual live endpoint. Returns either of the two JSON messages 
above along with the corrected module_code
\end{verbatim}

\section{Functional Requirements}\label{sec4:fr}
The final testing was conducted using a list of functional requirements set out in section \ref{sec1:fr} and the testing table can be found in Appendix \ref{tab:fr-test}. Most of the tests passed correctly however some of them have not.

The Functional requirements FR3.4 \& FR4.7 have only been partially fulfilled. This was a purposeful design choice as it would mean that the website would require that Discord servers are added to database and that they would have to be updated when any issues occurred. This was also bad because this list would have to be maintained and could not be automatically updated which would lead to it breaking down the line.