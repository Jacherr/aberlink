\chapter{Testing}

Detailed descriptions of every test case are definitely not what is required here. What is important is to show that you adopted a sensible strategy that was, in principle, capable of testing the system adequately even if you did not have the time to test the system fully.

Provide information in the body of your report and the appendix to explain the testing that has been performed. How does this testing address the requirements and design for the project?

How comprehensive is the testing within the constraints of the project?  Are you testing the normal working behaviour? Are you testing the exceptional behaviour, e.g. error conditions? Are you testing security issues if they are relevant for your project? 

Have you tested your system on ``real users''? For example, if your system is supposed to solve a problem for a business, then it would be appropriate to present your approach to involve the users in the testing process and to record the results that you obtained. Depending on the level of detail, it is likely that you would put any detailed results in an appendix.

Whilst testing with ``real users'' can be useful, don't see it as a way to shortcut detailed testing of your own. Think about issues discussed in the lectures about until testing, integration testing, etc. User testing without sensible testing of your own is not a useful activity.

The following sections indicate some areas you might include. Other sections may be more appropriate to your project. 

\section{Overall Approach to Testing}

\section{Automated Testing}

\subsection{Unit Tests}

\subsection{User Interface Testing}

\subsection{Stress Testing}

\subsection{Other types of testing}

\section{Integration Testing}

\section{User Testing}